% TEMPLATE FOR STANDARD QUIZ
% by laura

\documentclass[11pt,epsfig]{article}
\usepackage{amsmath} 
\usepackage{mdframed}
\usepackage{graphicx}
\usepackage{multirow}
\def\infinity{\rotatebox{90}{8}}
\oddsidemargin=0in
\evensidemargin=0in
\textwidth=6.3in
\topmargin=-0.5in
\textheight=9in
\renewcommand{\arraystretch}{2}
\parindent=0in
\pagestyle{empty}


%------------------------------------------------------------------
% PROBLEM, PART, AND POINT COUNTING...

% Create the problem number counter.  Initialize to zero.
\newcounter{problemnum}

% Specify that problems should be labeled with arabic numerals.
\renewcommand{\theproblemnum}{\arabic{problemnum}}


% Create the part-within-a-problem counter, "within" the problem counter.
% This counter resets to zero automatically every time the PROBLEMNUM counter
% is incremented.
\newcounter{partnum}[problemnum]

% Specify that parts should be labeled with lowercase letters.
\renewcommand{\thepartnum}{\alph{partnum}}

% Make a counter to keep track of total points assigned to problems...
\newcounter{totalpoints}

% Make counters to keep track of points for parts...
\newcounter{curprobpts}		% Points assigned for the problem as a whole.
\newcounter{totalparts}		% Total points assigned to the various parts.

% Make a counter to keep track of the number of points on each page...
\newcounter{pagepoints}
% This counter is reset each time a page is printed.

% This "program" keeps track of how many points appear on each page, so that
% the total can be printed on the page itself.  Points are added to the total
% for a page when the PART (not the problem) they are assigned to is specified.
% When a problem without parts appears, the PAGEPOINTS are incremented directly
% from the problem as a whole (CURPROBPTS).


%---------------------------------------------------------------------------


% The \problem environment first checks the information about the previous
% problem.  If no parts appeared (or if they were all assigned zero points,
% then it increments TOTALPOINTS directly from CURPROBPTS, the points assigned
% to the last problem as a whole.  If the last problem did contain parts, it
% checks to make sure that their point values total up to the correct sum.
% It then puts the problem number on the page, along with the points assigned
% to it.

\newenvironment{problem}[1]{
% STATEMENTS TO BE EXECUTED WHEN A NEW PROBLEM IS BEGUN:
%
% Increment the problem number counter, and set the current \ref value to that
% number.
\refstepcounter{problemnum}
%
% Add some vspace to separate from the last problem.
\vspace{0.15in} \par
%
\setcounter{curprobpts}{#1} \setcounter{totalparts}{0}	% Reset counters.
%
% Now put in the "announcement" on the page.
{\Large \bf \theproblemnum. \normalsize %({\it \arabic{curprobpts}
%{point\null\ifnum \value{curprobpts} = 1\else s\fi}\/)
}
}{
% STATEMENTS TO BE EXECUTED WHEN AN OLD PROBLEM IS ENDED:
%
% If no parts to problem, then increment TOTALPOINTS and PAGEPOINTS for the
% entire problem at once.
\ifnum \value{totalparts} = 0
	\addtocounter{totalpoints}{\value{curprobpts}}	% Add pts to total.
	\addtocounter{pagepoints}{\value{curprobpts}}	% Add pts to page total.
%
% If there were parts for the problem, then check to make sure they total up
% to the same number of points that the problem is worth. Issue a warning
% if not.
\else \ifnum \value{totalparts} = \value{curprobpts}
	\else \typeout{}
	\typeout{!!!!!!!   POINT ACCOUNTING ERROR   !!!!!!!!}
	\typeout{PROBLEM [\theproblemnum] WAS ALLOCATED \arabic{curprobpts} POINTS,}
	\typeout{BUT CONTAINS PARTS TOTALLING \arabic{totalparts} POINTS!}
	\typeout{}
	\fi
\fi
}


%---------------------------------------------------------------------------


% The \newpart command increments the part counter and displays an appropriate
% lowercase letter to mark the part.  It adds points to the point counter
% immediately.  If 0 points are specified, no point announcement is made.
% Otherwise, the announcement is in scriptsize italics.

\newcommand{\newpart}[1]
{
\refstepcounter{partnum}	% Set the current \ref value to the part number.
\hspace{0.25in}		% Indent the part by a quarter inch.
%
% If points are to be printed for this problem (signaled by point value > 0),
% then put them in in scriptsize italics.
\ifnum #1 > 0
	\makebox[0.5in][l]{{\bf \thepartnum.} {\bf ({\it #1 pt\ifnum #1 = 1\else s\fi\/}) \,\,}}
\else
	\makebox[0.25in][l]{({\bf \thepartnum})}
\fi
%
\hspace{0.1in}		% Lead the material away from the part "number".
%
\addtocounter{totalparts}{#1}	% Add points to totalparts for this problem.
\addtocounter{pagepoints}{#1}	% Add points to total for this page.
\addtocounter{totalpoints}{#1}	% Add points to total for entire test.
}


%---------------------------------------------------------------------------



% Just in case you want to skip some numbers in your test...

\newcommand{\skipproblem}[1]{\addtocounter{problemnum}{#1}}



%---------------------------------------------------------------------------


% The \showpoints command simply gives a count of the total points read in up to
% the location at which the command is placed.  Typically, one places one
% \showpoints command at the end of the latex file, just prior to the
% \end{document} command.  It can appear elsewhere, however.

\newcommand{\showpoints}
{
\typeout{}  
\typeout{====> A TOTAL OF \arabic{totalpoints} POINTS WERE READ.}
\typeout{}
}


%---------------------------------------------------------------------------



\begin{document}


%%%(change to appropriate class and semester)
\textbf{Revise(Chapter 5): CIS 2033 Sprint 2015: Computational Prob and Stat}

%%%(change to appropriate quiz type and date)
\textbf{Shanshan Zhang {(tuf14438@temple.edu)} }


% problem
\section{Notations}
$X, Y, Z$: Random Variables.\\
$k,a,b$: specific whole number (e.g. 0,1,2,...). \\
$x,y$: specific real number (e.g. 1.2, 0.5, ...). \\
$p(k)$: Probability mass function for discrete random variable $X$. It calculate for any specific whole number $k$, the probability of $P(X=k)$.\\
$f(x)$: Probability density function for continuous random variable $X$.\\
$F(x)=P(X\le x)$: Distribution function, sometimes it's called cumulative distribution function.\\
\begin{tabular}{|l|}
\hline
If $X$ is discrete: $F(k) = \sum_{y\le k}p(y)$, where $y$ is any possible value for $X$ that is less or equal than $k$.\\
If $X$ is continuous: $F(x)=\int	^x_{-\infinity}f(x)dx$\\
\hline
\end{tabular}
 
\subsection{Important distributions}
\begin{tabular}{ |l|l|l|p{6cm}| }
%\caption{Important Distributions}
\hline
& \bf{Notation }& \bf{$p(k)$ or $f(x)$} & \bf{$F(k)$ or $F(x)$}\\ \hline
\multirow{3}{*}{Discrete} & $X\sim Ber(p)$ & $p(1) = p;p(0)=1-p$ & $F(k)=0,(k<0);F(k)=1-p,(0\le k <1);F(k)=1,(k>1)$\\ \cline{2-4}
 & $X\sim Bin(n,p)$ & $p(k)={n\choose k} p^k(1-p)^{n-k},k=0,1,...,n$& $F(k)=\sum_{y\le k}p(y),k=0,1,...,n$\\  \cline{2-4}
 & $X\sim Geo(p)$ & $p(k)=(1-p)^{k-1}p,k=1,...$& $F(k)=1-(1-p)^{k},k=0,1,...$\\ \cline{2-4}
 & $X\sim Pois(\mu)$ & $p(k)=\frac{\mu^k}{k!}{e^{-k}},k=0,1,...$ & $F(k)=\sum_{y\le k}p(y),k=0,1,...$\\ \hline
\multirow{4}{*}{Continuous} & $X\sim Unif(\alpha,\beta)$ & $f(x)=\frac{1}{\beta-\alpha},for~x\in (\alpha,\beta)$& $F(x)=\frac{x-\alpha}{\beta-\alpha}, for~ x \in (\alpha,\beta)$\\ \cline{2-4}
 & $X\sim Exp(\lambda)$ & $f(x)=e^{-\lambda x}, for~ x \in [0,\infinity)$& $F(x)= 1 - e^{-\lambda x},for~ x \in [0,\infinity)$\\ \cline{2-4}
 & $X\sim Par(\alpha)$ & $f(x)=\frac{\alpha}{x^{\alpha+1}},for ~x\in [1,\infinity)$& $F(x)=1-\frac{1}{x^\alpha},for ~x\in [1,\infinity)$\\ \cline{2-4}
& $X\sim N(\mu,\sigma^2)$&$f(x)=\frac{1}{\sqrt{(2\pi\sigma^2)}}e^{-\frac{(x-\mu)^2}{2\sigma^2}},x\in (-\infinity,+\infinity)$&NO explicit form\\ \hline
\end{tabular}
\newpage
\section{Table of Basic Integrals}
\subsection{Basic Forms}


\begin{equation}
\int x^n dx = \frac{1}{n+1}x^{n+1},\hspace{1ex}n\neq -1
\end{equation}

\begin{equation}
\int \frac{1}{x}dx = \ln |x|
\end{equation}

\begin{equation}
\int u dv = uv - \int v du
\end{equation}

\begin{equation}
\int \frac{1}{ax+b}dx = \frac{1}{a} \ln |ax + b| 
\end{equation}

\subsection{Integrals of Rational Functions}

\begin{equation}
\int \frac{1}{(x+a)^2}dx = -\frac{1}{x+a}
\end{equation}

\begin{equation}
\int (x+a)^n dx = \frac{(x+a)^{n+1}}{n+1}, n\ne -1
\end{equation}

\begin{equation}
\int x(x+a)^n dx = \frac{(x+a)^{n+1} ( (n+1)x-a)}{(n+1)(n+2)}
\end{equation}

\begin{equation}
\int \frac{1}{1+x^2}dx = \tan^{-1}x
\end{equation}

\begin{equation}
\int \frac{1}{a^2+x^2}dx = \frac{1}{a}\tan^{-1}\frac{x}{a}
\end{equation}

\begin{equation}
\int \frac{x}{a^2+x^2}dx = \frac{1}{2}\ln|a^2+x^2|
\end{equation}

\begin{equation}
\int \frac{x^2}{a^2+x^2}dx = x-a\tan^{-1}\frac{x}{a}
\end{equation}

\begin{equation}
\int \frac{x^3}{a^2+x^2}dx = \frac{1}{2}x^2-\frac{1}{2}a^2\ln|a^2+x^2|
\end{equation}

\begin{equation}
\int \frac{1}{ax^2+bx+c}dx = \frac{2}{\sqrt{4ac-b^2}}\tan^{-1}\frac{2ax+b}{\sqrt{4ac-b^2}}
\end{equation}

\begin{equation}
\int \frac{1}{(x+a)(x+b)}dx = \frac{1}{b-a}\ln\frac{a+x}{b+x}, \text{ } a\ne b
\end{equation}

\begin{equation}
\int \frac{x}{(x+a)^2}dx = \frac{a}{a+x}+\ln |a+x|
\end{equation}


\begin{equation}
\int \frac{x}{ax^2+bx+c}dx = \frac{1}{2a}\ln|ax^2+bx+c| 
-\frac{b}{a\sqrt{4ac-b^2}}\tan^{-1}\frac{2ax+b}{\sqrt{4ac-b^2}}
\end{equation}
\end{document}