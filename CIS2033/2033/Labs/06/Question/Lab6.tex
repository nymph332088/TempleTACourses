% !TEX TS-program = pdflatex
% !TEX encoding = UTF-8 Unicode

% This is a simple template for a LaTeX document using the "article" class.
% See "book", "report", "letter" for other types of document.

\documentclass[11pt]{article} % use larger type; default would be 10pt

\usepackage[utf8]{inputenc} % set input encoding (not needed with XeLaTeX)

%%% Examples of Article customizations
% These packages are optional, depending whether you want the features they provide.
% See the LaTeX Companion or other references for full information.

%%% PAGE DIMENSIONS
\usepackage{geometry} % to change the page dimensions
\geometry{a4paper} % or letterpaper (US) or a5paper or....
% \geometry{margin=2in} % for example, change the margins to 2 inches all round
% \geometry{landscape} % set up the page for landscape
%   read geometry.pdf for detailed page layout information

\usepackage{graphicx} % support the \includegraphics command and options

% \usepackage[parfill]{parskip} % Activate to begin paragraphs with an empty line rather than an indent

%%% PACKAGES
\usepackage{booktabs} % for much better looking tables
\usepackage{array} % for better arrays (eg matrices) in maths
\usepackage{paralist} % very flexible & customisable lists (eg. enumerate/itemize, etc.)
\usepackage{verbatim} % adds environment for commenting out blocks of text & for better verbatim
\usepackage{subfig} % make it possible to include more than one captioned figure/table in a single float
% These packages are all incorporated in the memoir class to one degree or another...

\usepackage{color}
\usepackage{url}

%%% HEADERS & FOOTERS
\usepackage{fancyhdr} % This should be set AFTER setting up the page geometry
\pagestyle{fancy} % options: empty , plain , fancy
\renewcommand{\headrulewidth}{0pt} % customise the layout...
\lhead{}\chead{}\rhead{}
\lfoot{}\cfoot{\thepage}\rfoot{}

%%% SECTION TITLE APPEARANCE
\usepackage{sectsty}
\allsectionsfont{\sffamily\mdseries\upshape} % (See the fntguide.pdf for font help)
% (This matches ConTeXt defaults)

%%% ToC (table of contents) APPEARANCE
\usepackage[nottoc,notlof,notlot]{tocbibind} % Put the bibliography in the ToC
\usepackage[titles,subfigure]{tocloft} % Alter the style of the Table of Contents
\renewcommand{\cftsecfont}{\rmfamily\mdseries\upshape}
\renewcommand{\cftsecpagefont}{\rmfamily\mdseries\upshape} % No bold!

%%% END Article customizations

%%% The "real" document content comes below...

\title{Lab Six\\
Computational Probability and Statistics \\
CIS 2033, Section 002}
\author{Due: 9:00 AM, Friday, Dec. 05, 2014}
\date{} % Activate to display a given date or no date (if empty),
         % otherwise the current date is printed 

\begin{document}
\maketitle

\paragraph*{Question 1}
%Poiss
Suppose that the number of customers visiting a store in a day can be modeled as a Poisson process. Now, we have a dataset {\it poiss2.mat} \footnote{\url{http://astro.temple.edu/~tud09663/course/cps/poiss2.mat}}, which contains such yearly records for this store. Now, 
\begin{enumerate}
\item Please estimate the intensity $\lambda$ for this Poisson process;
\item How about the number of customers visiting this store in a week ? Please also estimate its intensity parameter $\hat{\lambda}$. 
\item Please generate 100 weekly records for this store. You can randomly generate 100 numbers from such a Poisson distribution, $Poiss(\hat{\lambda})$. Please save those records in a mat file (e.g., poiss100.mat). 
\end{enumerate}
 
\paragraph*{Question 2}
%Exp
Let the random variable $X$ follow an Exponential distribution such that $X \sim Exp(\lambda)$. Please download the dataset, {\it exp.mat} here \footnote{\url{http://astro.temple.edu/~tud09663/course/cps/exp.mat}}. This dataset contains 500 samples as the realization of $X$. Please empirically estimate $\lambda$ by using those samples. Let the random variable $Y = 2X$. What kind of specific distribution Y should have and how to compute its parameters. Based on the computed parameters, please randomly generate 100 samples as the realization of $Y$ from its distribution.  Please save those 100 samples into a mat file (e.g., exp100.mat). 

\paragraph*{Question 3}
%MLL
In this question, we will use maximum likelihood estimation to determine the intensity parameters, $\lambda$ for multiple Poisson processes. Please download the dataset, {\it poiss3.mat} \footnote{\url{http://astro.temple.edu/~tud09663/course/cps/poiss3.mat}}. In this dataset, there is an instance matrix $X$ with the size of $1000\times 4$. The $i$-th column of $X$ stores 1000 records for one Poisson process, $Poiss(\lambda_i)$, for $i=1,2,3,4$. Now, 
\begin{enumerate}
\item Please use maximum likelihood estimation to determine those parameters $\lambda_i, i=1, 2, 3, 4$.  
\item Based on the computed parameters, $\lambda_i, i=1, 2, 3, 4$, please compute the log-likelihoods based on those data samples. For the $i$-th column, compute the likelihood $L(\lambda_i)$ on those 1000 samples. 
\end{enumerate} 

\paragraph*{Question 4}
%MLL
In this question, we will use maximum likelihood estimation to determine the parameters, $\lambda$ for multiple Exponential distributions. Please download the dataset, {\it exp2.mat} \footnote{\url{http://astro.temple.edu/~tud09663/course/cps/exp2.mat}}. In this dataset, there is an instance matrix $X$ with the size of $500\times 3$. The $i$-th column of $X$ stores 500 records for one exponential distribution, $Exp(\lambda_i)$, for $i=1,2,3$. Now, 
\begin{enumerate}
\item Please use maximum likelihood estimation to determine those parameters $\lambda_i, i=1, 2, 3$.  
\item Based on the computed parameters, $\lambda_i, i=1, 2, 3$, please compute the likelihoods based on those data samples. For the $i$-th column, compute the log-likelihood $L(\lambda_i)$ on those 500 samples. 
\end{enumerate} 

\end{document}
